\documentclass[12pt,letterpaper]{article}
\usepackage{fullpage}
\usepackage[top=2cm, bottom=4.5cm, left=2.5cm, right=2.5cm]{geometry}
\usepackage{amsmath,amsthm,amsfonts,amssymb,amscd}
\usepackage{lastpage}
\usepackage{enumerate}
\usepackage{fancyhdr}
\usepackage{mathrsfs}
\usepackage{xcolor}
\usepackage{graphicx}
\usepackage{listings}
\usepackage{hyperref}

\hypersetup{%
  colorlinks=true,
  linkcolor=blue,
  linkbordercolor={0 0 1}
}

\setlength{\parindent}{0.0in}
\setlength{\parskip}{0.05in}
%%%%%%%%%%%%%%%%%%%%%%%%%%%%%%%%%%%%%%%%%%%%%%%%%%%%%%%%%% }

%%%%%%%%%%%%%%%%%%%%%%%% CHANGE HERE %%%%%%%%%%%%%%%%%%%% {
\newcommand\course{Mechatronics}
\newcommand\semester{Fall 2019}
\newcommand\hwnumber{}                 % <-- ASSIGNMENT #
\newcommand\NetIDa{Your Name: }           % <-- YOUR NAME
%%%%%%%%%%%%%%%%%%%%%%%%%%%%%%%%%%%%%%%%%%%%%%%%%%%%%%%%%% }

%%%%%%%%%%%%%%%%% DO NOT CHANGE HERE %%%%%%%%%%%%%%%%%%%% {
\pagestyle{fancyplain}
\headheight 35pt
\lhead{\NetIDa}
\lhead{\NetIDa}                 
\chead{\textbf{\Large C/C++ Examination \hwnumber}}
\rhead{\course}
\lfoot{}
\cfoot{}
\rfoot{\small\thepage}
\headsep 1.5em
%%%%%%%%%%%%%%%%%%%%%%%%%%%%%%%%%%%%%%%%%%%%%%%%%%%%%%%%%% }

\lstset { %
    language=C++,
    backgroundcolor=\color{white}, % set backgroundcolor
    basicstyle=\footnotesize,% basic font setting
}


%\title{C/C++ Exam}

\begin{document}

\begin{enumerate}
    \item Correct the error(s) in the following code:
    
    \begin{lstlisting}
#include <stdio.h>
#include <stdlib.h>

void fun(int* ptr) {
    int i = 0;
    int j = 0;
    int arrSize = 4;
    ptr = (int*) malloc(sizeof(int) * arrSize);
    
    for (i = 0; i < arrSize; i++) {
        ptr[i] = 1;
        for (j =0; j < arrSize; j++) {
            ptr[i % (j+1)] *= arrSize + ptr[i] - ptr[j];
        }
    }
}

int main() {
    int* data;
    fun(data);
    printf("The first element of the list is %i\n", *data);
    free(data);
}

    \end{lstlisting}
    
    \item Correct the error(s) in the following code:
    \begin{lstlisting}
#include <stdio.h>
#include <stdlib.h>

struct Node {
    struct Node* next;
    int val;
};

void prependNode(struct Node** head, struct Node* newNode) {
    newNode.next = *head;
    *head = newNode;
}

void cleanupList(struct Node* head) {
    if (head == NULL)
        return;
    cleanupList(head.next);
    free(head);
}

int main() {
    struct Node* root = malloc(sizeof(struct Node));
    for (int i = 0; i < 17; i++) {
        struct Node* node = malloc(sizeof(struct Node));
        node.val = i*i;
        prependNode(&root, node);
        printf("%i\n", root.val);
    }

    cleanupList(root);
}
    \end{lstlisting}
    
    \item What does the following code output:
    
    \begin{lstlisting}
#include <iostream>
#include <vector>

using namespace std;

static constexpr int SIZE = 5;

int main() {
    vector<int> vec;
    for (int i = 0; i <= SIZE; i++) {
        vec.emplace_back(i);
    }

    cout << vec[0] << endl;
    for (int i = 1; i < SIZE; i++) {
        vec[i] = 2*vec[i-1] + vec[i];
        cout << vec[i] << endl;
    }
}

    \end{lstlisting}
    
    \item Why should you \textbf{never} use the using namespace statement in a header file?
    
    \vspace{20mm}
    
    \item What is the function of \textbf{constexpr} and why is it better to use than \textbf{\#define}?
    
    \vspace{20mm}
    
    \item Given the vector2 ctypes structure defined in python, create a structure utilizing this called Matrix2by2.
    
    \lstset{language=python}
    \begin{lstlisting}
    from ctypes import *

    class VECTOR2(Structure):
        _fields_ = [("x1", c_int), ("x2", c_int)]
        
    \end{lstlisting}
    
    \vspace{20mm}
    
    \item Write a program that takes in three variables $a,b,c$ and finds the coefficients of the following multivariate polynomial $(ax + by + cz)^3$ with variables $x, y, z$.  Hint: Start with 
    $(ax+by+cz)^2$.  For bonus points, extend this to $(ax+by+cz)^n$.
    
\end{enumerate}

\end{document}

